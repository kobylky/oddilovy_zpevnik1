\beginsong{Lachtani}[by={Jaromír Nohavica}]
\beginrefrain
\chordson
\lrep\ \[C]Lach, lach, \[F]jé, \[C]jé  \[Ami]lach, lach, \[G]jé, \[C]jé. \rrep
\endrefrain

\beginverse
\chordson
\[C]Jedna lachtaní \[F]rodi\[C]na \[Ami]rozhodla se, že si vyjde \[G]do ki\[C]na,
jeli vlakem, lodí, metrem a pak \[F]tramva\[C]jí
a teď \[Ami]u kina Vesmír \[G]lachta\[C]jí,
\[G]lachtaní úspory \[C]dali dohro\[G]mady,
koupili si lístky \[C]do první \[G]řady,
\[C]otec lachtan řekl netře\[F]me bí\[C]du,
\[Ami]pro každého koupil pytlík \[G]araší\[C]dů.
\endverse
\emptyrefrain

\beginverse
\chordsoff
Na jižním pólu je nehezky, a tak lachtani si vyjeli na grotesky,
těšili se, jak bude veselo,
když zazněl gong a v sále se setmělo,
co to ale vidí jejich lachtaní zraky,
sníh a mráz a sněhové mraky,
pro veliký úspěch změna programu,
dnes dáváme film ze života lachtanů.
\endverse
\emptyrefrain

\beginverse
\chordsoff
Táta lachtan vyskočil ze sedadla, nevídaná zlost ho popadla,
„Proto jsem se netrmácel přes celý svět,
abych tady v kině mrznul jako turecký med,
tady zima, doma zima, všude je chlad,
kde má chudák lachtan relaxovat?“
Nedivte se té lachtaní rodině,
že pak rozšlapala arašidy po kině.
\endverse
\emptyrefrain

\beginspecialverse{C: }
\chordson
\[C]Tahle lachtaní \[F]rodi\[C]na
\[Ami]od té doby nechodí už \[G]do ki\[C]na, \[G]lach, \[C]lach.
\endspecialverse
\endsong