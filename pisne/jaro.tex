\beginsong{Jaro}[by={J. Šťáhlavský, T. Linka}]
\beginverse
\chordson
\[Ami]My čekali \[C]jaro a \[G]zatím přišel \[Ami]mráz,
tak strašlivou \[C]zimu ne\[G]poznal nikdo z \[Ami]nás,
z těžkých černých \[C]mraků se \[G]stále sypal \[Ami]sníh
a vánice \[C]sílí v po\[G]ryvech ledo\[Ami]vých.
\endverse

\beginverse*
\chordson
\[C]Z chýší dřevo mizí a \[G]mouka ubývá,
\[Dmi]do sýpek se raději už \[G]nikdo nedívá,
\[C]zvěř z okolních lesů nám \[G]stála u dveří
\[Dmi]a hladoví ptáci při\[G]létli za zvěří, a stále \[Ami]blíž.
\endverse

\beginverse
\chordsoff
Jednoho dne večer, to už jsem skoro spal,
když vystrašenej soused na okno zaklepal:
„Můj synek doma leží, v horečkách vyvádí,
já do města bych zašel, doktor snad poradí.“
\endverse

\beginverse*
\chordsoff
Půjčil jsem mu koně, a když sedlo zapínal,
dříve, než se rozjel, jsem ho ještě varoval:
„Nejezdi naší zkratkou, je tam příkrej sráz
a v týhletý bouři tam snadno zlámeš vaz, tak neriskuj!“
\endverse

\beginverse
\chordsoff
Na to strašný ráno dnes nerad vzpomínám,
na tu strašnou chvíli, když kůň se vrátil sám,
trvalo to dlouho, než se vítr utišil,
na sněhové pláně si každý pospíšil.
\endverse

\beginverse*
\chordsoff
Jeli jsme tou zkratkou až k místu, které znám,
kterým bych v té bouři nejel ani sám,
a pak ho někdo spatřil, jak tam leží pod srázem,
krev nám tuhla v žilách nad tím obrazem, já klobouk sňal.
\endverse

\beginspecialverse{C:}
\chordson
\[Ami]Někdy ten, kdo \[C]spěchá, se \[G]domů nevra\[Ami]cí~…
\endspecialverse
\endsong