\beginsong{Strom}[by={Jiřina Doležalová}]
\beginverse
\chordson
\[Ami]Polní cestou kráčeli \[G]šumaři do vísky hrát,
\[Ami]svatby, pohřby tahle cesta \[G]poznala mnohokrát,
po \[F]jedné svatbě se \[G]chudým lidem \[Ami]synek narodil
a \[F]táta mu u \[G]prašný cesty \[E]života strom zasadil.
\endverse

\beginrefrain
\chordson
A on tam \[A]stál a koukal \[F#mi]do polí,
byl jak \[D]král, sám v celém \[E]okolí,
korunu \[F#mi]měl, korunu měl, i když ne \[D]ze zlata
a jeho \[A]pokladem byla \[E]tráva střapa\[A]tá.
\endrefrain

\beginverse
\chordsoff
Léta běží a na ten příběh si už nikdo nevzpomněl,
jen košatý strom se u cesty ve větru tiše chvěl
a z vísky bylo město a to město začlo chtít,
asfaltový koberec až na náměstí mít.
\endverse
\emptyrefrain

\beginverse
\chordsoff
Že strom byl v cestě plánované, to malý problém byl,
ostrou pilou se ten problém snadno vyřešil,
tak naposled se do nebe náš strom pak podíval
a tupou ránu do větvoví už snad ani nevnímal.
\endverse
\emptyrefrain

\beginverse
\chordsoff
Při stavbě se objevilo, že silnice bude dál,
a tak kousek od nové cesty smutný pařez stál,
dětem a výletníkům z výšky nikdo nemával
a přítel vítr si na strništích z nouze o něm píseň hrál.
\endverse

\beginrefrain
\chordsoff
Jak tam stál a koukal do polí,
byl jak král, sám v celém okolí,
korunu měl, korunu měl, i když ne ze zlata,
a jeho pokladem byla tráva střapatá~…
\endrefrain
\endsong