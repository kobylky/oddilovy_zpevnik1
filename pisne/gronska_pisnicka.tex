\beginsong{Grónská písnička}[by={Jaromír Nohavica}]

\beginverse
\chordson
\[D]Daleko \[Emi]na severu \[A]je Grónská \[D]zem,
\[D]žije tam \[Emi]Eskymačka \[A]s Eskymá\[D]kem,
\lrep\ \[D]my bychom \[Emi]umrzli, jim \[G]není zi\[D]ma,
snídají \[Emi]nanuky \[A]a esky\[D]ma. \rrep
\endverse

\beginverse
\chordsoff
Mají se bezvadně, vyspí se moc,
půl roku trvá tam polární noc,
\lrep\ na jaře vzbudí se a vyběhnou ven,
půl roku trvá tam polární den. \rrep
\endverse

\beginverse
\chordsoff
Když sněhu napadne nad kotníky,
hrávají s medvědy na četníky,
\lrep\ medvědi těžko jsou k poražení,
neboť medvědy ve sněhu vidět není. \rrep
\endverse

\beginverse
\chordsoff
Pokaždé ve středu, přesně ve dvě
zaklepe na na íglů hlavní medvěd:
\lrep\ „Dobrý den, mohu dál na vteřinu?
Nesu vám trochu ryb na svačinu.“ \rrep
\endverse

\beginverse
\chordsoff
V kotlíku bublá čaj, kamna hřejí,
psi venku hlídají před zloději,
\lrep\ smíchem se otřásá celé íglů,
medvěd jim předvádí spoustu fíglů. \rrep
\endverse

\beginverse
\chordsoff
Tak žijou vesele na severu,
srandu si dělají z teploměrů,
\lrep\ my bychom umrzli, jim není zima,
neboť jsou doma a mezi svýma. \rrep
\endverse
\endsong