\beginsong{Náhrobní kamen}[by={Petr Novák}]
\beginverse
\chordson
\[G]Když půjdeš \[Emi]po cestě, \[G]kde růže \[A]vadnou,
kde rostou \[G]stromy \[C]bez lis\[G]tí,
\[G]tak vyjdeš \[Emi]na místo, \[G]kde tvý slzy \[A]spadnou
na \[G]hrob, \[C]co nikdo nečis\[G]tí.
\endverse

\beginverse
\chordsoff
Jen starej rozbitej náhrobní kámen,
řekne ti, kdo nemoh už dál.
Tak sepni ruce svý a zašeptej ámen
ať jsi tulák nebo král.
\endverse

\beginspecialverse{Rec.:}
\chordsoff
Dřív děvče chodilo s kyticí růží
rozdávat lidem štěstí a smích.
Oči jí maloval sám Bůh černou tuší,
pod jejím krokem tál sníh.
\endspecialverse

\beginverse
\chordsoff
Všem lidem dávala náručí plnou,
sázela kytky podél cest.
Jednou však zmizela a jako když utne,
přestaly růže náhle kvést.
\endverse

\beginverse
\chordsoff
Pak jsem ji uviděl, ubohou vílu,
na zvadlých květech věčně snít.
Všem lidem rozdala svou lásku a sílu,
že sama dál nemohla už žít.
\endverse

\beginverse
\chordsoff
Tak jsem jí postavil náhrobní kámen
a čerstvé růže jsem tam dal.
Pak jsem se pomodlil a zašeptal ámen
a svojí píseň jsem jí hrál.
\endverse
\endsong