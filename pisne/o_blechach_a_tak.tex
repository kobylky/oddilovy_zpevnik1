\beginsong{O blechách a tak}[by={Karel Plíhal}]
\beginverse
\chordson
Jsem starý \[Ami]penzion pro osamělé \[Emi]blechy,
ne každá \[Dmi]blecha má to štěstí, že má \[E]chotě,
celé dny \[Dmi]poslouchám ty jejich marné \[E]vzdechy,
všechny se \[Dmi]upíjejí \[D#dim]někde o sa\[E]motě.
\endverse

\beginverse
\chordsoff
Tak blecha Jindřiška přišla o svého druha,
v předvečer svatební to pro ni rána byla,
zapíjel svobodu a chlastal jako duha
a krevní destička mu v krku zaskočila.
\endverse

\beginrefrain
\chordson
\[G]Tyhle \[C]tragédie \[G]bleší vás \[Ami]jistě nepo\[Emi]těší,
buď\[F]me k nim trochu \[G]vlídněj\[C]ší, \[G]
vždyť i \[C]lidé krev si \[G]pijí a, \[Ami]koneckonců, \[Emi]žijí
a \[F]svět je stále \[G]krásněj\[E]ší.
\endrefrain

\beginverse
\chordsoff
A bleše Tamaře zas blešák s jinou zahnul,
jednou je přistihla, no prostě - žádná psina,
pod tíhou svědomí on na život si šáhnul,
skočil mi do dlaně, když fackoval jsem syna.
\endverse

\beginverse
\chordsoff
A blecha Renata je osamělá matka,
starej si odskočil na kolemjdoucí jehně,
to jehně zavedli na nedaleká jatka,
co nocí proplakala na mém levém stehně.
\endverse

\beginrefrain
\chordson
+ a svět je \[F]stále \[G]krásněj\[C]ší~… \[F] \[G] \[C]
\endrefrain
\endsong