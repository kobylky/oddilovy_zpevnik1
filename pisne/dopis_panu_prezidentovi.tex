\beginsong{Dopis panu prezidentovi}[by={Jaromír Nohavica}]
\beginverse
\chordson
Pane \[D]prezidente -- ústavní \[Emi]činiteli
píšu \[A7]Vám dopis že moje děti na mě \[D]zapomněly.
Jde o syna Karla a mladší \[Emi]dceru Evu,
rok už \[A7]nenapsali, rok už nepřijeli na \[D]návštěvu.
\endverse

\beginrefrain
\chordson
Vy to \[G]pochopíte, vy přece \[Ami]všechno víte,
vy se \[D]poradíte, vy to vyřešíte, vy mě \[G]zachráníte.
\chordsoff
Pane prezidente, já chci jen kousek štěstí,
pro co jiného jsme přeci zvonili klíčema na náměstí.
\endrefrain

\beginverse
\chordsoff
Pane prezidente a ještě stěžuju si,
že mi podražili pivo, jogurty, párky i trolejbusy,
i poštovní známky, i bločky na poznámky,
i telecí plecko, Řecko i Německo, no prostě všecko.
\endverse
\emptyrefrain

\beginverse
\chordsoff
Pane prezidente mé České republiky,
oni mě propustili na hodinu z mý fabriky.
Celých třicet roků všecko bylo v cajku,
teď přišli noví mladí a ti tu řádí jak na Klondiku.
\endverse
\emptyrefrain

\beginrefrain
\chordsoff
Vy to pochopíte vy se mnou soucítíte~…
\endrefrain

\beginverse
\chordsoff
Pane prezidente, moje Anežka kdyby žila,
ta by mě za ten dopis co tu teď píšu patrně přizabila.
Řekla by: Jaromíre chováš se jak malé děcko,
víš co on má starostí s celú tu Evropu s vesmírem a vůbec všecko.
\endverse

\beginrefrain
\chordsoff
Ale vy mě pochopíte~… + Pane prezidente.
\endrefrain
\endsong