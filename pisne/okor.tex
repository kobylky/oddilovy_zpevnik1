\beginsong{Okoř}[by={lidová}]
\beginverse
\chordson
\[D]Na Okoř je cesta jako žádná ze sta, \[A7]vroubená je stroma\[D]ma.
\[D]když du po ní v létě samoten ve světě, \[A7]sotva pletu noha\[D]ma.
\[G]Na konci té cesty \[D]trnité \[E]stojí krčma jako \[A7]hrad.
\[D]Tam zapadlí trampi hladoví a sešlí, \[A7]začli sobě noto\[D]vat.
\endverse

\beginrefrain
\chordson
\[D]Na hradě Okoři \[A7]světla už nehoří, \[D]bílá paní \[A7]šla už dávno \[D]spát.
\[D]Ta měla ve zvyku \[A7]podle svého budíku \[D]o půlnoci \[A7]chodit straší\[D]vat.
\[G]Od těch dob co jsou tam \[D]trampové \[E]nesmí z hradu \[A7]pryč.
\[D]A tak dole v podhradí \[A7]se šerifem dovádí,
\[D]on ji sebral \[A7]od komnaty \[D]klíč.
\endrefrain

\beginverse
\chordsoff
Jednoho dne z rána roznesla se zpráva, že byl Okoř vykraden.
Nikdo neví dodnes kdo to tenkrát odnes, nikdo nebyl dopaden.
Šerif hrál celou noc mariáš s bílou paní v kostnici.
Místo aby hlídal, zuřivě ji líbal, dostal z toho zimnici.
\endverse
\emptyrefrain
\endsong
